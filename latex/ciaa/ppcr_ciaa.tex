% http://icalp11.inf.ethz.ch/
% Submission Deadline: Feb 15, 2011
% 12 pages
% This is LLNCS.DEM the demonstration file of
% the LaTeX macro package from Springer-Verlag
% for Lecture Notes in Computer Science,
% version 2.3 for LaTeX2e

\documentclass{llncs}

%\usepackage[pdftex]{graphicx}
\usepackage{amsmath}
\usepackage{amsfonts}
\usepackage{makeidx}
\usepackage{listings}
\usepackage{algorithmic}
\usepackage{algorithm}
\usepackage{fancyvrb}

\newcommand{\Bs}[1]{\texttt{\symbol{92}}#1}
\newcommand{\B}[1]{\symbol{123}#1}
\newcommand{\BB}[1]{\symbol{125}#1}
\newcommand{\Dr}[1]{\symbol{36}#1}

\DefineVerbatimEnvironment%
  {VERBATIM}{Verbatim}
  {fontfamily=courier, fontsize=\small, commandchars=\\\{\}, numbers=left, xleftmargin=5mm}

\newcommand{\derives}{\stackrel{*}{\Longrightarrow}}

\begin{document}

\frontmatter          % for the preliminaries
%
\pagestyle{headings}  % switches on printing of running heads
\addtocmark{Extending LR Parsing Capabilities} % additional mark in the TOC
%
%\chapter*{Preface}
%
%\chapter*{Organization}
%
%\section*{Sponsoring Institutions}
%
%Bernauer-Budiman Inc., Reading, Mass.\\
%The Hofmann-International Company, San Louis Obispo, Cal.\\
%Kramer Industries, Heidelberg, Germany
%
%\tableofcontents
%
\mainmatter              % start of the contributions
%
\title{
Solving LR Conflicts via Nested Parsing and Dynamic Modification of The Action Table
%Extending LR Conflict Resolution Mechanisms%\footnotemark[3]
%Extending yacc Conflict Resolution Language%\footnotemark[3]
}
%

\titlerunning{A way simple to solve the conflicts in a LR parsing}
%                                     also used for the TOC unless
%                                     \toctitle is used
%
\author{
C.~Rodriguez-Leon and L. Garcia-Forte
}
%
\authorrunning{C.~Rodriguez-Leon and L. Garcia-Forte}
%
%%%% list of authors for the TOC (use if author list has to be modified)
\tocauthor{Ivar Ekeland, Roger Temam, Jeffrey Dean, David Grove,
Craig Chambers, Kim B. Bruce, Elisa Bertino}
%
\institute{
Departamento de EIO y Computaci\'on,\\
Universidad de La Laguna, Tenerife, Spain\\
\email{[casiano|lgforte]@ull.es},\\ WWW home page:
\texttt{http://nereida.deioc.ull.es}
}

\maketitle              % typeset the title of the contribution

\hyphenation{spe-cu-la-ti-ve spe-ci-fi-ca-tion li-mi-ta-tion
me-cha-nisms}

\begin{abstract}
%\begin{itemize}
%\item
%PPCR only
%\item new PPCR syntax
%\item
%Theoretical
%\item backtracking LR with PPCR
%\item general syntax
%\item
%nested parsing: how the table construction is modified
%\item examples
%\item
%experimental results
%
%\end{itemize}
Yacc-like LR parser generators provide ways to statically solve
shift-reduce mechanisms based on token precedence. No
mechanisms are provided for the resolution of difficult reduce-reduce
and shift-reduce conflicts. To solve such kind of conflicts the language
designer has to modify the grammar.  The usual approach
to cope with these limitations has been to extend
LR parsers with the capacity to branch at any multivalued
table entry. 
The strategy presented in this paper takes an entirely different path:
It extends
yacc conflict resolution sub-language with new dynamic
conflict resolution constructs specifying which production 
must be chosen in terms of the results of some nested parsing.
The LR parsing table is then modified at parsing time using the
information provided by the nested parser.
Through this simple modification LALR parsers
are able to parse ambiguous gramars and recognize inherently ambiguous
languages. 





\end{abstract}

\section{Introduction}
Yacc-like LR parser generators \cite{Johnsonyacc} provide ways to solve shift-reduce mechanisms
based on token precedence. No mechanisms are provided for the resolution
of difficult reduce-reduce or shift-reduce conflicts. 
To solve such kind of conflicts the language designer 
has to modify
the grammar. 

Some  context-dependency ambiguities can be solved through the use of lexical tie-ins: a flag
which is set by the semantic actions, whose purpose is to alter the way
tokens are parsed. 
A more general solution is to extend 
LR parsers with the capacity to branch
at any multivalued entry of the LR action table. For example, Bison \cite{bison},
via the \verb|%glr-parser directive| 
and Elkhound \cite{elkhound}
provide implementations of the Generalized LR (GLR) algorithm \cite{tomita2}.
In the GLR algorithm, when a conflicting transition is encountered, 
the parsing stack is forked
into as many parallel parsing stacks as conflicting actions.
The next input token
is read and used to determine the next transitions for each of the
top states. If some top state does not transit for 
the input token it means that path is invalid and that branch
is discarded. 
Though GLR has been successfully applied to
the parsing of ambiguous languages, the handling 
of languages that are both context-dependent and ambiguous 
is more difficult \cite{kelbt}. 
The user must always analyze
the conflicts to make sure that GLR splitting is only
done where it is intended. A GLR parser splitting inadvertently may cause
problems less obvious than an LALR parser statically choosing the wrong
alternative in a conflict. Furthermore, the interactions with the lexer
have to be considered with great care. Since a split parser consumes
tokens without performing any actions during the split, the lexer cannot
obtain information via parser actions. 

The strategy presented in Sections
\ref{section:ppcrforrr}
extends yacc conflict resolution mechanisms
with new ones, supplying ways to resolve conflicts
that can not be solved using static precedences. 
The algorithm for the generation of the LR tables is essentially the same
but the parsing tables can be modified
at run time. 

The technique involves labelling the points in conflict in the grammar specification
and providing additional information to resolve the conflict when it arises.
Crucially, this does not requires rewriting or transforming the grammar,
trying to resolve the conflict in advance, backtracking or branching into concurrent
speculative parsers. Instead, the resolution is postponed until the conflict actually arises
during parsing, whereupon some code inspects the state of the underlying parse
engine to decide the appropriate solution. 
Any grammar ambiguity can be fixed with a minimal amount of branching.

This technique can be combined to complement both GLR and backtracking LR
\mbox{algorithms} to give the programmer a finer control of the branching process.
It puts the user - as it naturally occurs in top down parsing - 
in control of the parsing strategy when the grammar is ambiguous,
making it easier to deal with efficiency and context dependency issues.

The Postponed Conflict Resolution  strategy (PPCR) does not 
requires more work or more knowledge of LR parsing
than the usual in yacc programming.
LR conflict removal is, however,  a laborious task.
The number of conflicts in a programming language can reach tens and even hundreds:
The original grammars of Algol-60 and Scheme
result in 61 and  78 conflicts respectively with 
an average density of one conflict for each two productions.
By adding Postponed Conflict Resolution to the classical  precedence and associativity 
settings we can fix all the conflicts in such grammars without modifying the grammars.
Removing conflicts while preserving the grammar is preferable to rewriting the grammar in 
several situations: 
When using a  conflict removal tool like the one described in
\cite{passos}, since the language designer will be still familiar with the resulting grammar,
when the original grammar better reflects  the author ideas about the syntax and semantic of the language,
when the original grammar is easier to read and to understand (size matters)
and when (see section \ref{section:inherentlyambiguous}) such unambiguous grammar is hard or impossible to find.


The techniques described in this paper have been implemented in 
\verb|eyapp| \cite{Rodriguez:Leon},
a yacc-like LALR parser generator for \mbox{Perl \cite{Wall:Perl}}. 
All the grammar examples used in this work can be found in \cite{lgforte}.
%The Perl language is, quoting Paul Hudak's \mbox{article \cite{hudak}} a ``{\it domain specific language for text manipulation}".



This paper is divided in four sections.
The next section introduces the use of the Postponed Conflict Resolution
strategy and 
illustrates it with several examples.
Section \ref{section:ppcrlr} explains the algorithms behind and 
some implementation details.
The last section 
summarizes the advantages and disadvantages of our proposals.
%
%



\section{The \textit{Postponed Conflict Resolution} Strategy}
\label{section:ppcrforrr}

The {\it Post\underline{p}oned Conflict Resolution} is a strategy (PPCR strategy)
to apply whenever there is a shift-reduce or reduce-reduce conflict which is
unsolvable using static precedences. It delays the decision, whether to shift
or reduce and by which production to reduce, to parsing time.
Let us assume the \verb|yacc|/\verb|eyapp| \mbox{compiler \cite{Rodriguez:Leon}}
announces  the presence of
a reduce-reduce conflict.
The steps followed to solve a reduce-reduce conflict using the
PPCR strategy are:

\begin{enumerate}
\item
Identify and understand the conflict: What LR(0)-items/productions and tokens are involved?.

Tools must support that stage, as for example via  the \verb|.output| file generated by \verb|eyapp| 
using option \verb|-v| or the graphical description obtained using option \verb|-w| or 
with the help of expert systems \cite{passos,basten}.
Suppose  we identify that the participants are two LR(0)-items \mbox{$A \rightarrow \alpha_\uparrow$}
and
$B \rightarrow \beta_\uparrow$ when the lookahead token is \verb|@|.

\item The software must allow the use of symbolic labels to refer
by name to the productions involved in the conflict.
Let us assume that production $A \rightarrow \alpha$
has label \texttt{:rA} and production $B \rightarrow \beta$
has label \texttt{:rB}.
A difference with \verb|yacc| is that
in \verb|Parse::Eyapp| productions can have {\it names} and {\it labels}.
In \verb|Eyapp| names and labels can be
explicitly given using the directive \verb|%name|, using a syntax similar to this one:

\begin{eqnarray*}
\mbox{{\tt \%name :rA}}& A &\rightarrow \alpha\\
\mbox{{\tt \%name :rB}}& B &\rightarrow \beta 
\end{eqnarray*}


\item
Give a symbolic name to the conflict. In this case we choose \verb|isAorB|
as name of the conflict.

\item
\label{item:mark}
Inside the {\it body} section of the grammar, mark the points of conflict using the
new reserved word \verb|%PREC| followed by the conflict name:

\begin{eqnarray*}
\mbox{{\tt \%name :rA\ }}    A &\rightarrow \alpha &\mbox{\tt\ \%PREC IsAorB }\\
\mbox{{\tt \%name :rA\ }}    B &\rightarrow \beta  &\mbox{\tt\ \%PREC IsAorB }
\end{eqnarray*}
\item
\label{item:handler}
Introduce a \verb|%conflict| directive inside the {\it head} section of
the translation scheme to specify the way the conflict will be solved.
The directive is followed by some code - known as the {\it conflict handler} -
whose mission is to modify
the parsing tables. This code will be executed
each time the associated conflict state is reached.
This is the usual layout of the conflict handler:

\begin{center}
\begin{tabular}{p{8cm}}
\begin{verbatim}
%conflict  IsAorB {
  if ($is_A) { $self->YYSetReduce(':rA' ); } 
       else { $self->YYSetReduce(':rB' ); }
}
\end{verbatim}
\end{tabular}
\end{center}

Inside a conflict code handler the Perl default variable \verb|$_| refers to
the input and \verb|$self| refers to the parser object.

%\begin{quotation}
Variables in Perl - like \verb|$self| -
have prefixes like \verb|$| (scalars),
\verb|@| (lists), \verb|%| (hashes or dictionaries), \verb|&| (subroutines), etc.
specifying the type of the variable. These prefixes are called {\it sigils}.
The sigil \verb|$| indicates a {\it scalar} variable, i.e. a variable that stores a single value:
a number, a string or a reference. In this case \verb|$self| is a reference to the parser object.
The arrow syntax \verb|$object->method()| is used to call a method: it is the equivalent
of the dot operator \verb|object.method()| used in most
OOP languages. Thus the call

\begin{center}
\begin{tabular}{p{5cm}}
\begin{verbatim}
$self->YYSetReduce(':rA' ) 
\end{verbatim}
\end{tabular}
\end{center}

is a call to the
\verb|YYSetReduce| method of the object \verb|$self|.
%\end{quotation}

The method \verb|YYSetReduce| provided by \verb|Parse::Eyapp| receives a 
production label, like \verb|:rA|. The former call
sets the parsing action
for the state associated with the
conflict \verb|IsAorB| to
reduce by the production \verb|:rA| for the tokens involved in the conflict.

\item
The value \verb|$is_A|
represents the context-dependent
dynamic knowledge that allows us to take the right
decision.
It is usually the result of a call to a nested parser of a subgrammar component 
of the main grammar, but it can also be any other contextual information we have
to determine which one is the right production. 

The general procedure is 
to identify suitable {\it exploration points} in the right hand side of some productions.
They are  marked inside
the grammar body using the \verb|%conflictname?| syntax:
\begin{center}
\begin{tabular}{p{5cm}}
\begin{VERBATIM}[numbers=none,codes={\catcode`$=3\catcode`\_=8\catcode`^=7}] 
C: $\lambda_1$ \textbf{%AorB?} $\lambda_2$
\end{VERBATIM}
\end{tabular}
\end{center}
The set of exploration points must hold the following covering property:
any visit to the conflictive state must be preceded by a visit to 
an exploration state. The exploration states have associated {\it an exploration
code} which is given (explicitly or implicitly) through a \verb|%explorer| directive.
Its goal is to determine which production applies.
That directive is set inside the head section, and often follows this pattern (see section 
\ref{section:simpleexample}
for an example):
\begin{center}
\begin{tabular}{p{6.2cm}}
\begin{VERBATIM}[numbers=none, codes={\catcode`$=3\catcode`\_=8}]
\textbf{%explorer AorB D}
%% /* body section */
\ldots
\end{VERBATIM}
\end{tabular}
\end{center}
where \verb|D| is an auxilary syntactic variable.
The former directive is translated by the \verb|eyapp|
compiler into  the following exploration code:
\begin{center}
\begin{tabular}{p{5.8cm}}
\begin{VERBATIM}[numbers=none, codes={\catcode`$=3\catcode`\_=8}]
\Dr{}is\_A = \textbf{\Dr{}self->YYPreParse('D')}; 
\end{VERBATIM}
\end{tabular}
\end{center}
The call \verb|$self->YYPreParse(D)|
returns true if, and only if there is a prefix of the unexpended input
that belongs to the language defined by \verb|D|. 
The language defined by \verb|D|
must characterize the \verb|$is_A| condition: any incoming 
input belongs to the language if, and only if, 
$A \rightarrow \alpha_\uparrow$ is a suitable the handler
for the anti-derivation.


%but does not exist a derivation $\lambda_2 \derives \nu_1 B \nu_2$
\end{enumerate}




%\section{Low Level and High Level PPCR}
%\label{section:nestedparsing}
%% change it to 
%\label{section:amb}

\subsection{An Example: The C++ ambiguity}
This section illustrates the technique through a problem 
that arises in C++.
The C++ syntax \cite{c++} does not disambiguate between expression
statements (\verb|stmt|) and declaration statements (\verb|decl|). 
The ambiguity arises when an expression
statement has a function-style cast as its left-most subexpression.
Since C \cite{c} does not support function-style casts, this ambiguity does not occur
in C programs.
For example, the phrase \verb|int (x) = y+z;| 
parses as either a \texttt{decl} or a \texttt{stmt}.
The disambiguation rule used in C++ is that {\it
if the statement can be interpreted both as a declaration and
as an expression, the statement is interpreted as a declaration statement.
}
The following examples  disambiguate into {\it expression} statements when the
potential {\it declarator} is followed by an operator different from equal 
or semicolon (\texttt{type\_spec} stands for a type specifier):

\begin{center}
\begin{tabular}{|p{3.5cm}|p{4.5cm}|}
\hline
expr  & dec\\
\hline
\begin{verbatim}
type_spec(i)++;      
type_spec(i,3)<<d;  
type_spec(i)->l=24;
\end{verbatim}
&
\begin{verbatim}
type_spec(*i)(int); 
type_spec(j)[5];   
type_spec(m) = { 1, 2 }; 
\end{verbatim}\\
\hline
\end{tabular}
\end{center}

Regarding to this problem, Bjarne Stroustrup \cite{stroustrup} remarks:
\begin{quote}
\begin{it}
Simple lexical lookahead can help a parser disambiguate
most cases. Consider analyzing a statement consisting of
a sequence of tokens as follows:
\begin{verbatim}
              type_spec (dec_or_exp) tail
\end{verbatim}
Here \verb|dec_or_exp| must be a declarator, an expression, 
or both for the statement to be legal. This implies that \verb|tail|
must be a semicolon, something that can follow a 
parenthesized declarator or something that can follow
a parenthesized expression, that is, an initializer, \verb|const|,
\verb|volatile|, \verb|(|, \verb|[|, or a postfix or infix operator.
The general cases cannot be resolved without backtracking, nested grammars
or similar advanced parsing strategies. In particular,
the lookahead needed to disambiguate this case is not limited.
\end{it}
\end{quote}

\subsection{Simplifying the Problem}

The following Eyapp grammar 
depicts an oversimplified version of the C++ ambiguity 

\begin{center}
\begin{tabular}{p{8.5cm}}
\begin{VERBATIM}
%right '='
%left '+'
%%
prog: \textit{/* empty */}         | prog stmt ;
stmt: expr ';'            | decl      ;
expr: ID                  | NUM 
    | INT '(' expr ')'    \textit{# typecast} 
    | expr '+' expr       | expr '=' expr 
;
decl: INT declarator ';'  
    | INT declarator '=' expr ';' 
;
declarator: ID            | '(' declarator ')' ;
%%
\end{VERBATIM}
\end{tabular}
\end{center}


\subsection{Identifying the problem}

This grammar is ambiguous since an input like: \verb|int (x) = 4;|
can be interpreted as a \texttt{decl} or an \texttt{expr}.
The \texttt{eyapp} compiler warn us of the presence of reduce/reduce conflict:

\begin{verbatim}
$ eyapp -W SimplifiedCplusplusAmbiguity.eyp
  1 reduce/reduce conflict
\end{verbatim}

%\begin{figure}[htb]
%\includegraphics[scale=0.15]{SimplifiedCplusplusAmbiguity.png}
%\label{figure:cplusplussimplifiedpng}
%\caption{The LALR automaton for the simplified C++ grammar}
%\end{figure}

When we look at the \texttt{.output} file  or to the generated automaton graph
%(Figure \ref{figure:cplusplussimplifiedpng})
we see the reasons for the conflict:

\begin{VERBATIM}
Warnings:
---------
1 reduce/reduce conflict
Conflicts:
----------
................................... 
\textbf{State 18 contains 1 reduce/reduce conflict}
................................... 
State 18:

    \textbf{expr -> ID .    (Rule 5)}
    \textbf{declarator -> ID .  (Rule 12)}

    \textbf{')' [reduce using rule 12 (declarator)]}
    \textbf{')' reduce using rule 5 (expr)}
    '+' reduce using rule 5 (expr)
    '=' reduce using rule 5 (expr)
\end{VERBATIM}


This information tell us that when parsing 
\verb|'int (x|$_{\uparrow}$\verb|) = 4;'|,
once the parser has seen the \texttt{ID} \verb|x| and is in the presence
of the closing parenthesis \texttt{')'}, it is incapable to decide whether 
to reduce by rule 12 (\verb|declarator -> ID|) or rule 5 (\verb|expr -> ID|) .

%The disambiguation rule in C++ is: 
%{\it Take the phrase as a declaration if it looks as a declaration,
%otherwise is an expression.}


\subsection{Low Level PPCR Syntax}
\label{subsection:llpcr}

The next figure shows a low level solution
to the problem using PPCR. 
\begin{VERBATIM}
%\B{}
my $ISDEC;                   \label{vrb:isdec}
%\BB{}
%token NUM = /(\Bs{d}+)/
%token INT = /(int)\Bs{b}/
%token ID  = /([a-zA-Z_][a-zA-Z_0-9]*)/
%right '='          \label{vrb:left}
%left '+'           \label{vrb:right}
\textbf{%conflict decORexp} \B{}                  \label{vrb:conflictbegin}
  if ($ISDEC)                                     \label{vrb:isdecused}
    \B{} \textbf{$self->YYSetReduce('ID:DEC' )}; \BB{}
  else 
    \B{} \textbf{$self->YYSetReduce('ID:EXP' )}; \BB{}   \label{vrb:elseconflict}
\BB{}                                             \label{vrb:conflictend}
\textbf{%explorer decORexp} \B{} $ISDEC = \textbf{$self->YYPreParse('decl')}; \BB{} \label{vrb:explorer}
\textbf{%expect-rr 1}  \textit{# expect 1 reduce-reduce conflict} \label{vrb:expectrr}
%%
prog: \textit{/* empty */}
    | prog \textbf{%decORexp?} stmt
;
stmt: expr ';'
    | decl
;
expr:
      \textbf{%name ID:EXP}
      ID                    \textbf{%PREC decORexp} 
    | NUM
    | INT '(' expr ')' \textit{/* typecast */}
    | expr '+' expr
    | expr '=' expr
;
\textbf{decl}: INT declarator ';'        \label{vrb:decl}
    | INT declarator '=' expr ';'
;
declarator:
      \textbf{%name ID:DEC}
      ID                    \textbf{%PREC decORexp} 
    | '(' declarator ')'
;
%%
\end{VERBATIM}

Additionally to the classic static precedence directives (lines \ref{vrb:left}-\ref{vrb:right}) 
we can see two PPCR directives: \verb|%conflict| (lines \ref{vrb:conflictbegin}-\ref{vrb:conflictend})
and \verb|%explorer| (line \ref{vrb:explorer}). 

The auxiliary variable \verb|$ISDEC| 
(declared at line \ref{vrb:isdec} and used inside the conflict handler at line \ref{vrb:isdecused})
will be set to \verb|true| if, and only if, the conflictive incoming input
conforms (line \ref{vrb:explorer}) to the language defined by \verb|decl| (line \ref{vrb:decl}).

The conflict handler \verb|decORexp| simply sets the LR action
to reduce by 
\mbox{\tt expr $\rightarrow$} \mbox{ID}
or
\verb|declarator| $\rightarrow$ \verb|ID|
according to the value of \verb|$ISDEC| (lines \ref{vrb:isdecused}-\ref{vrb:elseconflict}).


The \verb|%explorer| directive used in line \ref{vrb:explorer} has the syntax:
\begin{center}
\begin{tabular}{p{147.43pt}}
\verb|%explorer conflictName { CODE }| 
\end{tabular}
\end{center}
the purpose of \verb|CODE| is to perform some nested 
parsing, to decide which action is correct.

The call to the method \verb|YYPreParse| in 
line \ref{vrb:explorer}
returns true if there is a prefix of the unexpended input
that belongs to the language defined by the \verb|decl| subgrammar (line \ref{vrb:decl}).

The point where the exploration starts - defining the time when the code 
of the \verb|%explore| directive is called -
is marked inside
the grammar body using the \verb|%conflictname?| syntax:

\begin{center}
\begin{tabular}{p{5cm}}
\begin{VERBATIM}[firstnumber=26]
prog:
    \textit{/* empty */}
  | prog \textbf{%decORexp?} stmt
;
\end{VERBATIM}
\end{tabular}
\end{center}

The \verb|eyapp| compiler 
creates a syntactic variable  
whose only empty production has as associated semantic action
the code defined in the \verb|%explorer| directive. The points where the 
\verb|%decORexp?| directive appears are substituted by that variable. 
Thus, the former lines are conceptually equivalent to:

\begin{center}
\begin{tabular}{p{8cm}}
\begin{VERBATIM}[numbers=none]
prog:
    \textit{/* empty */}
  | prog \textbf{decORexp.explorer} stmt
;

decORexp.explorer : \textit{/* empty */}
       { $ISDEC = $self->YYPreParse('decl'); }
\end{VERBATIM}
\end{tabular}
\end{center}

The \verb|%expect-rr 1| directive at line \ref{vrb:expectrr}
keeps \verb|eyapp| silent, 
avoiding the warnings regarding the reduce-reduce conflict.

\subsection{High Level PPCR Syntax}
\label{subsection:hlppcr}
The methodology followed in the previous section is so general that we 
have extended \verb|eyapp| with  a higher level syntax 
which specifies this particular way of 
solving difficult conflicts:
\begin{verbatim}
   %conflict conflictName nestedParser? actionName: actionName
\end{verbatim}
Where \verb|conflictName| is the name given to the conflict,
\verb|nestedParser| refers to the sub-grammar used for pre-parsing the incoming input
and \verb|actionName| is the name of a production involved in the conflict
or the reserved word \verb|shift|. 

The construct is translated as follows:
The parser for
\verb|nestedParser| is called each time the exploration point is reached, 
saving the result inside an attribute of the parser.
Each time the conflict state is reached, this attribute is checked 
and the corresponding parsing action is taken.

The next figure shows the complete header of the solution to the C++ example
using the new syntax.
The body is exactly the same that appears in the previous listing in section
\ref{subsection:llpcr}. In this version any explicit code has disappeared.

\begin{center}
\begin{tabular}{p{10.3cm}}
\begin{VERBATIM}
%token NUM = /(\Bs{d}+)/
%token INT = /(int)\b/
%token ID  = /([a-zA-Z_][a-zA-Z_0-9]*)/

%right '='
%left '+'

\textbf{%conflict decORexp decl? ID:DEC : ID:EXP}

%expect-rr 1  # expect 1 reduce-reduce conflict

%%
prog: \textit{/* empty */} | prog \textbf{%decORexp?} stmt ;
stmt: expr ';'    | decl ;
expr:
      \textbf{%name ID:EXP}
      ID                    \textbf{%PREC decORexp} 
    | NUM           | INT '(' expr ')' 
    | expr '+' expr | expr '=' expr
;
\textbf{decl}: INT declarator ';' | INT declarator '=' expr ';'
;
declarator:
      \textbf{%name ID:DEC}
      ID                    \textbf{%PREC decORexp} 
    | '(' declarator ')'
;
%%
\end{VERBATIM}
\end{tabular}
\end{center}


%\section{What is the Power of PPCR?}
%\label{section:xsx}
%In terms of grammar parsing abilities, we presume that any Context Free Grammar
can be parsed using a Low Level PPCR LALR parser to produce the AST the user requires.
This presumption is based in the fact that we 
can resort to the full Turing Complete target language to find
the conditions that satisfactorily solve the conflict. 
The answer to this question when using only the High Level PPCR syntax 
- the syntax introduced in section \ref{subsection:hlppcr} - is not so clear.
Here the expression {\it High Level PPCR} refers to PPCR 
solutions in which no explicit code in the hosting language appears,
either in the main grammar or in the auxiliar subgrammar used to decide the correct action.

We believe, but have no formal proof, that
the following unambiguous grammar 
is  an 
example of a Low Level PPCR grammar that is  not a High Level PPCR grammar.

\begin{center}
\begin{tabular}{p{4.2cm}}
\begin{verbatim}
%%
T: S          ;
S: x S x | x  ;
x: NUM   |  x OP NUM ;
%%
\end{verbatim}
\end{tabular}
\end{center}
Though it is straighforward to find equivalent LL(1) and LR(1) grammars\footnote{The language 
is even regular in {\tt x}. Is given by: {\tt /x(xx)*/}} the grammar can not be parsed by any LL(k) nor LR(k),  
nor by packrat parsing algorithms.  \cite{ford}.
An LR parser has to shift  on \verb|NUM| until the middle \verb|x| is reached. At that time it
has to reduce by $S \rightarrow x$. This is the \verb|eyapp| description of the conflict:  
\begin{center}
\begin{tabular}{p{9.3cm}}
\begin{VERBATIM}[numbers=none]
State 6:
 \textbf{S -> x . S x      (Rule 2)}
 \textbf{S -> x .          (Rule 3)}
 x -> x . OP NUM   (Rule 5)
 \textbf{NUM shift, and go to state 4}  \textit{# not the middle x}
 OP shift, and go to state 8
 $end reduce using rule 3 (S)
 \textbf{NUM [reduce using rule 3 (S)]} \textit{# is the middle x}
 S go to state 7
 x go to state 6

State 4:
	x -> NUM .	(Rule 4)
	$default	reduce using rule 4 (x)
\end{VERBATIM}
\end{tabular}
\end{center}
When state 6 is reached we have just seen an \verb|x| (observe how the three kernel items
have the dot after the \verb|x|). The presence of an \verb|OP| does not causes
conflict, since it means we must stay processing the current \verb|x|.
The presence of a \verb|NUM| indicates the start of a new \verb|x|.
We have to choose to reduce by the production \verb|S| $\rightarrow$ \verb|x| if
this \verb|x| is the one in the middle, otherwise we make a \verb|shift| 
to state 4 where the \verb|NUM| is reduced to a \verb|x|, starting the 
processing of the new arithmetic expression \verb|x|.
The challenge  here
is to make the parser work {\it without changing} the grammar structure.

Let us see that this grammar is Low Level PPCR. We indentify the conflict
- which we name \verb|isIn|\verb|TheMiddle| -, labelling as \verb|:MIDx| 
the reduction item and marking the exploration point:

\begin{center}
\begin{tabular}{p{8.6cm}}
\begin{VERBATIM}
%token NUM = /(\Bs{d}+)/
%token OP  = /([-+*\Bs{/}])/
%\B{}
my $nxs = 0;
%\BB{}
\textbf{%conflict isInTheMiddle} \B{}
  $nxs++;
  if ($nxs == \textbf{$self->YYVal('ExpList')}) \B{} 
    \textbf{$self->YYSetReduce(':MIDx' )};
    $nxs = 0; 
  \BB{}
  else \B{} \textbf{$self->YYSetShift()} \BB{} 
\BB{}  
\textbf{%explorer isInTheMiddle ExpList}

%%
T: \textbf{%isInTheMiddle?} S ; 
S:   x  \textbf{%PREC isInTheMiddle} S x  
  |  \textbf{%name :MIDx} 
     x  \textbf{%PREC isInTheMiddle} 
;
x: NUM | x OP NUM ;
%%
\end{VERBATIM}
\end{tabular}
\end{center}
The exploration code uses the following auxiliary
grammar \verb|ExpList|, which has been augmented with {\it semantic actions},
to compute the middle \verb|x|:
\begin{VERBATIM}
  %%
  \textbf{ExpList}: S    \B{} 1+int($_[1]/2) \BB{} ; 
  S:       S x  \B{} $_[1] + 1 \BB{} |  x   \B{} 1 \BB{} ;
  x:       NUM                |  x OP NUM ;
  %%
\end{VERBATIM}
The semantic value returned by the \verb|ExpList| grammar is later accessed inside the conflict handler
code above via the method call \verb|$self->YYVal('ExpList')|.

The conflict solver code is quite simple:
it keeps the position of the current \verb|x|
inside the variable \verb|$nxs|. Each time a new \verb|x| is seen,
\verb|$nxs| is incremented.
The reduction is called when the middle point is reached.

The proof that the conflicts in this grammar can not be solved using High Level PPCR
remains open for us.



%\section{Context Dependent Parsing}
%\label{section:simpleexample}
%The following
example
%\footnote{For the full examples used in this paper, see the directory {\tt examples/debuggingtut/} in the {\tt Parse::Eyapp} distribution \cite{Rodriguez:Leon}} 
accepts lists of two kind of commands:
{\it arithmetic expressions} like \verb|4-2-1| or one of two {\it associativity commands}:
\verb|left| or  \verb|right|. When a \verb|right| command is issued, the semantic of the
\verb|'-'| operator is changed to be right associative. When a \verb|left| command is issued
the semantic for \verb|'-'| returns to its classic left associative interpretation.
Here follows an example of input. Between shell-like comments appears the expected output:
\begin{center}
\begin{tabular}{p{8cm}}
\begin{verbatim}
$ cat input_for_dynamicgrammar.txt 
2-1-1 # left:  0 = (2-1)-1
RIGHT
2-1-1 # right: 2 = 2-(1-1)
LEFT
3-1-1 # left:  1 = (3-1)-1
RIGHT
3-1-1 # right: 3 = 3-(1-1)
\end{verbatim}
\end{tabular}
\end{center}
We use a variable \verb|$reduce| (initially set to \verb|1|)
to decide the way in which the ambiguity
\verb|NUM-NUM-NUM| is solved. If \verb|false| we will set the \verb|NUM-(NUM-NUM)|
interpretation. The variable \verb|$reduce|
is modified each time the input program emits a \verb|LEFT| or
\verb|RIGHT| command.

Following the steps outlined above,
and after looking at the \verb|.output| file, we see
that the items involved in the announced shift-reduce conflict are

\begin{eqnarray*}
expr \rightarrow expr_\uparrow - expr \\
expr \rightarrow expr - expr_\uparrow 
\end{eqnarray*}

\noindent and the lookahead token is \verb|'-'|.
We next mark the points in conflict in the grammar using
the \verb|%PREC| directive (see Listing \ref{listing:expressions})
\begin{lstlisting}[
        caption={ An Example of Context Dependent Ambiguity Resolution },
        label = {listing:expressions},
        frame=bottomline,
        keywords={PREC, my,return,lexer,tree,bypass,name,Elements,Int,String,YYParse,LexerGen,Gen},
        %firstnumber=1, 
        %frame=single,
        %numberstyle=\tiny,
        %numbers=right
        ]
%%
p:  
      /* empty */     {}
    | p c             {}
;

c:
      $expr { print "$expr\n" }
    | RIGHT { $reduce = 0}
    | LEFT  { $reduce = 1}
;

expr: 
      '(' $expr ')'  { $expr } 
    | %name :M
      expr.left         %PREC lOr
        '-' expr.right  %PREC lOr
         { $left -$right }

    | NUM
;
\end{lstlisting}

The $dollar$ and $dot$ notation used in some right hand sides (rhs)  like in \verb|expr.left| \verb|'-'| \verb|expr.right|
and \verb|$expr| is used to associate variable names with the attributes of the symbols.

The conflict handler \verb|lOr| defined in the header section
is:
\begin{lstlisting}[
        %caption={ An Example of Context Dependent Ambiguity Resolution },
        %label = {listing:expressions},
        frame=none,
        keywords={conflict,YYSetReduce,YYSetShift,
        PREC, if, else, my,return,lexer,tree,bypass,name,Elements,Int,String,YYParse,LexerGen,Gen},
        %firstnumber=1, 
        %frame=single,
        %numberstyle=\tiny,
        %numbers=right
        ]
%conflict lOr {
  if ($reduce) {$self->YYSetReduce('-', ':M')} 
  else         {$self->YYSetShift('-')}
}
\end{lstlisting}
If \verb|$reduce| is \verb|true|
we set the parsing action to {\it reduce} by the production labelled \verb|:M|,
otherwise we choose the {\it shift action}.

Observe how PPCR allow us to dynamically change at will the meaning of the same statement.
That is certainly harder to do  using alternative techniques,
either problem specific, like {\it lexical Tie-Ins} \cite{bison}, or more general,
like \mbox{GLR \cite{tomita2}}.





\section{Implementation and Performance}
\label{section:ppcrlr}
\label{alg:parser}       
For PPCR to work we need some slight modifications in the LR parsing algorithm.
The following Perl pseudocode describes the PPCR parsing algorithm. 
Our notation follows the conventions 
used in \cite{aho2006}:

\begin{VERBATIM}[codes={\catcode`$=3\catcode`\_=8\catcode`^=7}]
method PPCRparse(LRParser \Dr{}p)          \textit{# The LR parser object}
 my @stack;                            \textit{# The LR parsing stack}
 my \Dr{}s0 = \Dr{}p->startstate;
 push(@stack, \Dr{}s0);       \textit{# Push the start state in the stack}
 my \Dr{}b = \Dr{}p->yylex();                        \textit{# Get next token}
 forever do \B{}
   my \Dr{}s = top();     \textit{# Get the state in the top of the stack}
   my \Dr{}a = \Dr{}b;
   my \Dr{}act; 
   \textbf{if (\Dr{}p->isConflictive(\Dr{}s, \Dr{}a)) \Dr{}p->conflictHandler(\Dr{}s, \Dr{}a); } \label{vrb:conflictbegin}
   \Dr{}act = \Dr{}p->action(\Dr{}s->state, \Dr{}a);  \label{vrb:action}
   switch (\Dr{}act) \B{}
     case "shift t" : 
       my \Dr{}t;
       \Dr{}t->\B{}state\BB{} = t;
       \Dr{}t->\B{}attr\BB{}  = \Dr{}a->\B{}attr\BB{};
       push(\Dr{}t);               \textit{# Store the state in the stack}
       \Dr{}b = \Dr{}p->yylex();                     \textit{# Get next token}
       break;
     case "reduce A $\rightarrow \alpha$" : 
       my \Dr{}r;
                   \textit{# Get the attributes of the |$\alpha$| top states}
       my @ats = getAttributes(@stack[-|$\alpha$| .. -1]); 
       \Dr{}r->\B{}attr\BB{} = \Dr{}p->Sem\B{}A->$\alpha$\BB{}->(@ats); \label{vrb:semcall}
       pop(|$\alpha$|);                       \textit{# Pop length($\alpha$) states}
       my \Dr{}t = top();
       \Dr{}r->\B{}state\BB{} = \Dr{}p->goto(\Dr{}t, A); \label{vrb:goto}
       push(\Dr{}r); 
       break;
     case "accept" : return (\Dr{}s->attr); 
     default : \Dr{}p->yyerror("syntax error");
   \BB{}
\end{VERBATIM}

As usual \verb#|#$\alpha$\verb#|# denotes the length of string $\alpha$.
The call \verb|top()| returns the state in the top of the stack,
while the call \verb|pop(k)| extracts the top \verb|k| states from the stack.
The notations \verb|$a->{attr}| and \verb|$t->{attr}| make reference to the
attribute associated with the token  \verb|$a| and to the attribute associated with the state 
\verb|$s|.
The call \verb#getAttributes(@stack[-|#$\alpha$\verb#| .. -1])# returns the list of attributes
of the \verb#|#$\alpha$\verb#|#  top states in the stack. A negative index in Perl starts
from the end of the array, thus \verb|$stack[-1]| is the state in the top of the stack.
The hash entry \verb|$Sem{A|$\rightarrow \alpha$\verb|}| is a reference to the semantic code 
associated with production \verb|A|$\rightarrow \alpha$. 
Such code is called in line \ref{vrb:semcall},
whenever a reduction by \verb|A|$\rightarrow \alpha$ is detected. 


The calls to \verb|action| (line \ref{vrb:action}) and \verb|goto| (line \ref{vrb:goto}) 
respectively return the corresponding entries
of the \verb|ACTION| and \verb|GOTO| parsing tables.

The only change is in line \ref{vrb:conflictbegin}.
If the couple state-token is conflictive, i.e. if the LR action table was multivalued, the 
conflict handler for such entry is called to decide the correct action. 

This give us a measure 
of the overhead introduced by PPCR per iteration: the cost of checking the condition (line \ref{vrb:conflictbegin}). In case 
of being a conflictive state we also have to add the cost of executing the handler, 
which usually takes constant time: it checks the attribute containing the
result of the last explorer execution and changes accordingly the corresponding entry of the action table. 
Furthermore, a yacc compiler can generate optimal code by using
the classical LR algorithm when no conflict handlers were defined.

The correspondence between conflictive pairs $(state, token)$ 
and conflict handlers
is generated by the compiler during the construction of the LR tables using the information gathered from
the \verb|%PREC| directives.

An additional modification is needed when generating the LR tables for the sub-grammar starting in $A$.
Nested parsers for a nonterminal $A$ must accept prefixes  of the incoming input not necessarily
terminated by the end of input token. The action table is modified to have an $accept$ action 
for each entry $(s, a)$
where the state $s$ has an LR item $[A' \rightarrow A_\uparrow , a]$ where $a \in FOLLOW(A)$.
That is, it accepts when something derivable from $A$ has been seen and the next token is a legal one.
Here $A'$ denotes the super-start symbol for the sub-grammar starting in $A$. The set $FOLLOW(A)$ 
is the set of tokens $b$ in the super-grammar such that exists a derivation 
$S  \stackrel{*}{\Longrightarrow}  \alpha A b \beta$.
The end of input is also included in $FOLLOW(A)$ if, and only if,
exists a derivation $S \stackrel{*}{\Longrightarrow} \alpha A$.

For each explorer definition the current implementation of the \verb|eyapp| compiler 
creates an auxiliary syntactic variable  
whose only empty production has as associated semantic action
the code defined by the \verb|%explorer| directive. Such code can be explicit 
(see \cite{lgforte} for examples)
or implicit as in the example in section
\ref{section:inherentlyambiguous} 
. The points where the 
\verb|%conflict?| directive appears inside the body
are substituted by that variable. 
The execution of the exploration code is an additional source of overhead.
The programmer must choose the exploration point in a way that 
\begin{itemize}
\item
makes easy the nested parsing and the recollection of the information needed to take the decision
\item
minimizes the number of times it is executed and
the time taken by the decision/exploration code,
\item
guarantees that the decision taken will be valid when the next visit to the conflict state
occurs 
and
\item
does not introduce new conflicts. 
This restriction can be lifted if, instead of implementing explorers 
through a syntactic variable, the PPCR algorithm is modified to
execute the exploring code when the state is an explorer-state.
\end{itemize}

The last version of \verb|eyapp| can be obtained from  \cite{Rodriguez:Leon}.
There is also a repository of difficult and conflictive yacc grammars with their corresponding
PPCR solutions in \cite{lgforte}. Contributions are welcome.



%\section{Tokens Depending on the Syntactic Context}
%\label{section:pli}
%Compilers begin processing an input file by performing lexical analysis and syntactic analysis.
Lexical analyzers break the input in a sequence of units known as tokens.
Each token is representative of an easy to parse language: in fact, most of the time such languages can be represented
by regular expressions.
Usually there is a one-to-one relation between a token and a single regular expression. 
Problems arise,
however when a token's type depends upon contextual information \cite{schrodingerstoken}.
An example of this problem comes from PL/I, where statements like this are
legal:

\begin{verbatim}
         if then=if then if=then
\end{verbatim}
or even worse

\begin{verbatim}
         if then=if then if if=then then then=if
\end{verbatim}

The lexical analyzer has to determine which uses of \verb|if| and \verb|then| 
correspond to keywords and which correspond to identifiers.

In PL/I this problem arises because keywords like \verb|if| are not reserved and can be used in
other contexts. This same problem arises in many domain specific
languages \cite{schrodingerstoken}. 
The following Eyapp grammar \verb|PL_I_conflictNested| \verb|.eyp|
summarizes the problem:

\begin{lstlisting}[
        %caption={Setting the Generator}, 
        %label={listing:lexergen}, 
        frame=none,
        keywords={my,return,lexer,tree,bypass,name,Elements,Int,String,YYParse,LexerGen,Gen},
        %firstnumber=1, 
        %frame=single,
        %numberstyle=\tiny,
        %numbers=right
        ]
 1 %lexer { ... }
 2 
 3 %tree bypass
 4 %%
 5 stmt: ifstmt 
 6   | assignstmt
 7 ;
 8 
 9 ifstmt: 
10      %name IF
11     'if' expr 'then' stmt
12 ;
13 
14 assignstmt: 
15     %name ASSIGN
16     id '=' expr
17 ;
18 
19 expr:
20     %name EQ
21     id '=' id
22   | id
23 ;
24 
25 id: 
26     %name ID
27     ID
28 ;
29 
30 %%
\end{lstlisting}
The syntax is similar to \verb|yacc|.
The \verb|%lexer| directive at line 1 allow us
to define an explicit lexical analyzer. By default,
\verb|eyapp| automatically generates a lexical analyzer
from the 
\lstinline[
keywords={token,tree,bypass,name,Elements,Int,String,YYParse,LexerGen,Gen},
]
|%token NAME=/regexp/| definitions.

A difference with \verb|yacc| is that
in \verb|Parse::Eyapp| productions can have {\it names} and {\it labels}. 
In \verb|Eyapp| names and labels can be 
explicitly given using the directive \verb|%name|.
Names are used for different purposes.
When \verb|eyapp| builds a node of the abstract syntax tree
it uses the production name to decide the class 
of the node.

The code below 
shows the complete definition of the lexical analyzer:
\begin{lstlisting}[
        %caption={Setting the Generator}, 
        %label={listing:lexergen}, 
        frame=none,
        keywords={my,return,lexer,self,and,if,do,eq,
        self,YYExpects,YYPreParse,tree,bypass,name,Elements,Int,String,YYParse,LexerGen,Gen},
        %firstnumber=1, 
        %frame=single,
        %numberstyle=\tiny,
        %numbers=right
        ]
 1 %lexer {
 2   m{\G\s*(\#.*)?}gc;
 3 
 4   m{\G([a-zA-Z_]\w*)}gc and do {
 5     my $id = $1;
 6 
 7     return ('then','then')  if
 8          ($id eq 'then')        && 
 9          $self->YYExpects('then');
10     return ('if', 'if') if 
11          ($id eq 'if')          && 
12          $self->YYExpects('if') &&
13          !$self->YYPreParse('Assign');   
14     return ('ID', $id);
15   };
16 
17   m{\G(.)}gc and return ($1, $1);
18 
19   return('',undef);
20 }
\end{lstlisting}

The regular expression matchings 
at lines 2, 4 and 17 act against the parser 
input. 
When evaluated in a boolean context,
they return \verb|true| if the incoming input
matches the regular expression. 
Inside a \verb|%lexer| directive,
the default variable \verb|$_| refers to the 
input and the predefined
variable \verb|$self| refers to the parser object.

White spaces and bash-like comments are skipped in line 2.
The \verb|\G| anchor inside the regular expression 
matches the current input position.

The regular expression at line 4 detects the presence of
an identifier. 
Inside a regular expression, the bracketing construct 
\mbox{\tt(\ldots)}
creates {\it capture buffers}. To refer to the current contents of a
buffer we use \verb|$1| for the first parenthesis, \verb|$2|
for the second, and so on. The assignment at line 5
stores the first capture inside the lexically scoped variable \verb|$id|.


The token is \verb|then| whenever 
the identifier is \verb|then|
and the token \verb|then| is expected (lines 7-8) by
the syntax analyzer.
The problem can be solved using Eyapp \verb|YYExpects| method
which give us the full list of tokens that can be expected in a given 
parsing position. This way the scanner 
can have a more adaptive behavior fulfilling the syntax analyzer expectations.

More difficult is the analysis of the token \verb|if|.
Tokens \verb|if| and \verb|ID| can both appear after 
the \verb|then| token and consequently may both
appear inside the list of tokens returbed by \verb|YYExpects|.

Therefore, the token \verb|if| is returned
if the identifier is \verb|if| (line 11), 
\verb|if| $\in$ \verb|YYExpects| (line 12)
and the incoming input does not conform to a
assignment statement (line 13). 
Otherwise
the token \verb|ID| is returned (line 14).

The call \verb|$self->YYPreParse('Assign')|
parses the incoming input according to the 
\verb|Assign| subgrammar:

\begin{verbatim}
%token ID = /[a-zA-Z_]\w*/

%%
assignstmt: id '=' expr ;
expr: id '=' id | id ;
id: ID ;
%%
\end{verbatim}

One character tokens like \verb|'='| are handled by
the regular expression at line 17.

To get an executable, we compile the auxiliary grammar:

\begin{verbatim}
$ eyapp -P Assign.eyp
\end{verbatim}
The \verb|-P| option used when compiling \verb|Assign.eyp|
tells \verb|eyapp| to produce parsing tables that will accept 
if a prefix of the input belongs to the language generated\footnote{By default, a LR parser only accepts if the full input conforms 
to the grammar}.
We then proceed to compile the full grammar:
\begin{verbatim}
$ eyapp -C PL_I_conflictNested.eyp
\end{verbatim}

Option \verb|-C| tells \verb|eyapp| to
generate an executable instead of a Class. 
The \verb|eyapp| compiler provides a default \verb|main|
which will be used if no other is provided.
The default \verb|main| accepts several command line arguments:

\begin{verbatim}
$ ./PL_I_conflictNested.pm -t -i \
   -c 'if if=then then then=if'
\end{verbatim}
Option \verb|-t| tells the \verb|main| to print the result returned by the parser:
a  description of the syntax tree will be printed. Option \verb|-i| produces
information for the terminals in the abstract syntax tree.
Option \verb|-c| is followed by the input
for the parser. It indicates that the input is given in the command line.
The execution of the former command produces the following output describing
the abstract syntax tree:
\begin{verbatim}
IF(EQ(ID[if],ID[then]),ASSIGN(ID[then],ID[if]))
\end{verbatim}

The \verb|-t| option can be complemented with the \verb|-m| option.
The last must be followed by an integer.
It controls the display layout of 
the abstract syntax tree:
\begin{verbatim}
$ ./PL_I_conflictNested.pm -t -i -m 1 \
   -c 'if if=then then if if=if then then=if'
IF(
  EQ(
    ID[if],
    ID[then]
  ),
  IF(
    EQ(
      ID[if],
      ID[if]
    ),
    ASSIGN(
      ID[then],
      ID[if]
    )
  )
)
\end{verbatim}




%\section{Computing the Expected Tokens}
%\label{section:expected}
%One of the key contributions that made possible solving the PL-I
reserved-word versus identifier problem was 
the fact that the lexical analyzer can 
compute the \underline{exact} list of tokens 
expected by the syntax analyzer at any point of the scanning process.
This section describes the proposed  algorithm 
to compute this set. 
The algorithm uses simulation on the stack -
also known as symbolic interpretation - to achieve its goal.

The algorithm starts copying the parsing stack and callying the 
parser method $YYExpected$ with a reference to the parser stack:

%\IF {$i\geq maxval$} \STATE $i\gets 0$ \ELSE \IF {$i+k\leq maxval$} \STATE $i\gets i+k$ \ENDIF \ENDIF 
\begin{algorithmic}
\STATE    $YYExpected(parser\ stack)$
\end{algorithmic}

The pseudo-code for $YYExpected$ appears in Algorithm
\ref{algorithm:YYExpected}. In this and the next algorithms 
we will follow these conventions:
\begin{itemize}
\item $Q$ is the set of the states of the LR automaton
\item $\Sigma$ is the set of tokens or grammar terminals
\item $V$ is the set of variables or non-terminals
\item $P$ is the set of grammar productions
\item $ACTION$ denotes the LR action table: 
\[ ACTION : Q \times \Sigma \rightarrow Q \cup P \]
Where the kind of action is either a shift or a reduce:
\begin{eqnarray*}
ACTION(q, a) &=& shift\ q' \in Q\\
ACTION(q, a) &=& reduce\ by\ A \rightarrow \alpha \in P
\end{eqnarray*}

\item $GOTO$ is the LR automaton transition table restricted
to $V$:
\begin{eqnarray*}
GOTO : Q \times V &\rightarrow& Q\\
GOTO(q, A) &=& \ q' \in Q\\
\end{eqnarray*}
\end{itemize}

\begin{algorithm}[h] 
\caption{$YYExpected(\mathcal{S})$}
\label{algorithm:YYExpected}
\begin{algorithmic}%[1]
\REQUIRE {$Stack\ \mathcal{S}$}
\ENSURE The set of all expected tokens $\mathcal{E}$
\STATE $s = top(\mathcal{S})$
\STATE  $\mathcal{E} = \{ a \in \Sigma : \exists t \in Q\ such\ that\ ACTION(s, a) = shift\ t\ \}$
\STATE  $\mathcal{R} = \{  A \rightarrow \alpha \in P : ACTION(s, b) = reduce  A \rightarrow \alpha \}$
\IF {$\mathcal{R} \neq \emptyset$} 
  \STATE $\mathcal{E} = \mathcal{E} \cup YYSimStack(\mathcal{S}, \mathcal{R})$
\ENDIF
\RETURN $\mathcal{E}$
\end{algorithmic}
\end{algorithm}
Algorithm \ref{algorithm:YYExpected}  starts traversing the action table for the
state $s$ in the top of the stack. Tokens $a \in \Sigma$ 
for which there is a shift action 
are directly pushed in the set of expected 
tokens $\mathcal{E}$. 
The next step is to compute the set $\mathcal{R}$ of productions $A \rightarrow \alpha$  in
$P$ for which there
is a reduction. The remaining expected tokens
are computed calling $YYSimStack$, whose code appears in
Algorithm \ref{algorithm:YYSimStack}.


\begin{algorithm}[h]
\caption{$YYSimStack(\mathcal{S}, \mathcal{R})$}
\label{algorithm:YYSimStack}
\begin{algorithmic}[1]
\REQUIRE {$Stack\ \mathcal{S}, Productions\ \mathcal{R}$}
  \FORALL {$A \rightarrow \alpha \in \mathcal{R}$}
    \IF{$length(S) > length(\alpha)$}
      \STATE $Stack\ \mathfrak{S} = \mathcal{S}$
      \STATE \label{line:pop} $pop\ \mathfrak{S},\  length(\alpha)$
      \STATE $s = top(\mathfrak{S})$
      \STATE $n = GOTO(s, A)$
      \STATE $push\ \mathfrak{S}, n$
      \STATE \label{line:rec}$\mathcal{E} = \mathcal{E} \cup YYExpected(\mathfrak{S})$
    \ENDIF
 \ENDFOR
 \RETURN $\mathcal{E}$
\end{algorithmic}
\end{algorithm}

Algorithm \ref{algorithm:YYSimStack} 
consists of a simulation recursive process that 
branches at each available reduction.
The exploration is kept until no more 
productions can be applied for reduction.

The simulation mimics the LR parsing algorithm:
For each production $A \rightarrow \alpha \in \mathcal{R}$
the step of reducing by it is performed using a local stack 
$\mathfrak{S}$: As many states as the number of symbols in  the right hand side 
$\alpha$ are extracted from the stack (line \ref{line:pop}), leaving some state $s$ at the top.
The $GOTO$ table of the LR parser is then consulted to find which the 
next state $n$ is. The algorithm recursively calls back (line \ref{line:rec}) to 
$YYExpected$ %(Algorithm \ref{algorithm:YYExpected})
 to compute
the set of tokens expected in state $n$. The new tokens are added
to $\mathcal{E}$.


%\section{Other Applications: Data Generation Using LR parsers}
%\label{section:datageneration}
%The fact that we can compute the exact set of expected tokens 
has several applications. One of them is that a LR parser 
can be easily re-used to build a generator of the language described by
a grammar.

The idea  of grammar based data generation languages
has been exploited before \cite{maurer,yagg}.
What is new in the approach presented here is that
the same parser can be reused to 
generate data that is not only syntactically correct but
that also holds specified semantic restrictions. 
This section illustrates the proposed methodology
through the implementation of a generator of semantically correct
sequences of assignment statements.

\subsection{The Lexical Generator}

For each token the programmer defines a random generator,
which generates elements in the language associated with the token.
Then, instead of the typical lexical analyzer,
we use the lexical generator described in
Algorithm \ref{algorithm:lexicalgenerator}:


\begin{algorithm}[h] 
\caption{$LexicalGenerator(parser)$}
\label{algorithm:lexicalgenerator}
\begin{algorithmic}[1]
\REQUIRE {$LR\ Parser:\ parser$}
\ENSURE A syntactically correct pair $(token, attribute)$
\STATE  \label{line:weights} $weights = parser.weights()$
\STATE  \label{line:expected}$expected = parser.YYExpected()$
\STATE  \label{line:calltoFreq} $token = Freq(weights, expected)$
\STATE  \label{line:setgen}$generator = parser.generator(token)$
\STATE  \label{line:setattr}$attribute = generator.generate()$
\RETURN \label{line:returntoken}$(token, attribute)$
\end{algorithmic}
\end{algorithm}
    
The attribute $parser.weights$ of the $parser$ object
contains a probability distribution
which can be changed on the fly by the grammar semantic actions.
The call to $Freq$ at line \ref{line:calltoFreq}
randomly picks one of the expected tokens
according to the $weights$ probability distribution.
Once the $token$ is chosen among the expected ones ($parser.YYExpected()$), 
the lexer 
calls the $generate$ method of the corresponding generator 
$parser.generator(token)$ to compute the $attribute$.
Finally, the pair $(token, attribute)$ is returned to the
syntactic generator (line \ref{line:returntoken}).

Listings 
\ref{listing:lexergen}, 
\ref{listing:parsergen1}, 
\ref{listing:parsergen2}
and 
\ref{listing:usinggen}
describe a generator of syntactically and semantically
correct lists of assignment statments: Only initialized variables 
are used inside the generated expressions. 
The generated expressions are also free 
of overflow/underflow and division-by-zero errors.

Line 1 in Listing \ref{listing:lexergen} creates a syntax generator
object according to the grammar described in \verb|$package|.
The call to method \verb|LexerGen| in lines 2-21 
builds and set a lexical generator that behaves 
as described in Algorithm \ref{algorithm:lexicalgenerator}.

\begin{lstlisting}[
        caption={Setting the Generator}, 
        label={listing:lexergen}, 
        frame=bottomline,
        keywords={if, my, return,Elements,Int,String,YYParse,LexerGen,Gen}
        %firstnumber=1, 
        %frame=single,
        %numberstyle=\tiny,
        %numbers=right
        ]
 1 my $gen = $package->new();
 2 $gen->LexerGen(
 3  NUM => [ 2, Int(range=>[0, 9], sized=>0)],
 4  VAR => [
 5   0, # no variables are defined at the start
 6   Gen {
 7    if (keys %st) {
 8     return Elements(keys %st)->generate 
 9    }
10    return Int(range=>[0,9],sized=>0)->generate;
11   },
12  ],
13  VARDEF => [ 2,  String( length=>[1,2], 
14                          charset=>"A-NP-Z", 
15                          size => 100 )
16  ],
17  '=' => 2, '-' => 1, '+' => 2, 
18  '*' => 4, '/' => 2, '^' => 0.5, 
19  ';' => 1, '(' => 1, ')' => 2, 
20  ''  => 2, 'error' => 0,
21 );
\end{lstlisting}

%The call to the method \verb|LexerGen| (lines 2-21)
%dynamically generates a lexical generator subroutine that follows 
%the scheme described in algorithm \ref{algorithm:lexicalgenerator}.

The  method \verb|LexerGen| 
receives as arguments the set of pairs 
\[token \Rightarrow generator\ descriptor\]
A $generator\ descriptor$  is a pair 
\[[frequency\ weight, generator\ function]\]
as in:
\begin{lstlisting}[
        %caption={Setting the Generator}, label={listing:lexergenvardef}, 
        frame=none,
        keywords={Int,String,YYParse,LexerGen}
        %firstnumber=1, 
        %frame=single,
        %numberstyle=\tiny,
        %numbers=right
        ]
13  VARDEF => [ 2,  String( length=>[1,2], 
14                          charset=>"A-NP-Z", 
15                          size => 100 )
16  ],
\end{lstlisting}
which associates an absolute frequency and
a generator with the token \verb|VARDEF|. 
The call to function \verb!String! 
(See \cite{testlectrotest})
returns a function
which is a string generator: it generates
strings of length between 1 and 2 using the specified
char set.
A $generator\ descriptor$ can also be just a number
describing the frequency weight as in the case of the token \verb|'='|:

\begin{lstlisting}[
        %caption={Setting the Generator}, label={listing:lexergeneq}, 
        frame=none,
        keywords={Int,String,YYParse,LexerGen}
        %firstnumber=1, 
        %frame=single,
        %numberstyle=\tiny,
        %numbers=right
        ]
17  '=' => 2, '-' => 1, '+' => 2, 
\end{lstlisting}
In such case the corresponding generator 
is a \verb|Unit| 
(See \cite{testlectrotest})
generator: it always return the same 
value. That is, the generator for \verb|'='| always return \verb|'='|.

The token \verb|VAR| stands for variable identifiers.
We associate a weight 0 to it,
since at the beginning no variables have been
initialized. This inhibits the generation of the token \verb|VAR| by
the lexical generator:
\begin{lstlisting}[
        %caption={Setting the Generator}, label={listing:lexergeneq}, 
        frame=none,
        keywords={Elements,Int,String,YYParse,LexerGen, Gen}
        %firstnumber=1, 
        %frame=single,
        %numberstyle=\tiny,
        %numbers=right
        ]
 4  VAR => [
 5   0, # no variables are defined at the start
 6   Gen {
 7    if (keys %st) {
 8     return Elements(keys %st)->generate 
 9    }
10    return Int(range=>[0,9],sized=>0)->generate;
11   },
12  ],
\end{lstlisting}

The method \verb|Gen| receives as argument a subroutine
reference and builds a generator object from it.
The variable \verb|%st| used in lines 7 and 8 holds the program symbol table.
Its keys are the variable identifiers that have been 
already initialized. The associated generator (lines 7-10) consults
if the table is not empty. If so,
the call to \verb|Elements(keys %st)| returns a
generator that randomly chooses one of the identifiers in the
hash table \verb|%st|. Otherwise,
an integer number is returned (line 10) instead.

Observe how both tokens \verb|VAR| and \verb|VARDEF| stands
for the same lexical description: identifiers.
As in the former PL-I example in Section \ref{section:pli},
there is no ambiguity nor conflict. The problem is solved 
due to the close relation via \verb|YYExpected| between the lexer generator
and the syntax generator.
The places where \verb|VARDEF| is expected do not intersect
those in which \verb|VAR| is expected.

\subsection{The Syntax and Semantic Generator}
Once the lexical generator is set,
we proceed to define the syntax generator
using the grammar specification.
For each rule

\begin{center}
$A \rightarrow \alpha$ \verb|{ action() }|
\end{center}

\noindent the associated semantic action \verb|{ action() }|
works as a \verb|generate| method for $A \rightarrow \alpha$.
The semantic action can also be used to modify
the token probability distribution.
The bottom-up tree traversing of the abstract syntax tree
builds the final generated expression. 

Listings 
\ref{listing:parsergen1}
and
\ref{listing:parsergen2}
show the rest of the grammar. 
For most of the productions $A \rightarrow X_1 ... \ldots X_n$
the generator simply consists of the concatenation
of the attributes/phrases generated for the symbols $X_i$ in
their right hand side. The eyapp directive \verb|%defaultaction| 
sets the default semantic action (Listing \ref{listing:parsergen1}). 
At line 2 we extract a reference to the generator
object from the list of arguments in \verb|@_|. Line 3 
joins/concatenates the remaining arguments which are the
strings generated for the  $X_i$ symbols in the right hand side.

\begin{lstlisting}[caption={Setting the Default Semantic Action}, label={listing:parsergen1}, 
        frame=bottomline,
        keywords={defaultaction,Paste,use, base, Gen, sub, pushdeltaweight,popweight,Unit,
                  my,generate,while,do, return,join}
        %firstnumber=1, 
        %frame=single,
        %numberstyle=\tiny,
        %numbers=right
        ]

 1	%defaultaction {
 2	  my $self = shift;
 3	  return join '', @_;
 4	}
 5	
 6	%%
\end{lstlisting}

By setting the default action to \verb| join '', @_|
we can omit the semantic action in many of the grammar productions
as in:

\begin{lstlisting}[%caption={}, label={listing:parsergen1}, 
        %frame=bottomline,
        %keywords={defaultaction,Paste,use, base, Gen, sub, pushdeltaweight,popweight,Unit,
        %          my,generate,while,do, return}
        %firstnumber=1, 
        %frame=single,
        %numberstyle=\tiny,
        %numbers=right
        ]
exp:
    NUM                
  | VAR
  | exp '+' exp        
  | exp '-' exp        
  | exp '*' exp        
  | exp '/' exp        
  | exp '^' exp        
      ...
;

%%
\end{lstlisting}

There are, however, circumstances in which we want to modify on the fly
the kind of phrases generated by changing the token probability distribution.
One of those cases corresponds to the parenthesis rule 
\mbox{\tt expr $\rightarrow$ '(' expr ')'}:
\begin{lstlisting}[%caption={}, label={listing:parsergen1}, 
        %frame=bottomline,
        keywords={defaultaction,Paste,use, base, Gen, sub, pushdeltaweight,popweight,Unit,
                  my,generate,while,do, return}
        %firstnumber=1, 
        %frame=single,
        %numberstyle=\tiny,
        %numbers=right
        ]
 1 exp:
 2     NUM                
 3   | ... # other productions 
 4   | '('   { 
 5             $_[0]->pushdeltaweight(
 6               '(' => -1, 
 7               ')' => +1, 
 8               '+' => +1, 
 9             ); 
10           } 
11       exp 
12     ')'
13       {
14          $_[0]->popweight; 
15          "($_[3])"
16       }
17 ;
\end{lstlisting}
Each time an open parenthesis is generated, the intermediate semantic action
calls the parser method \verb|pushdeltaweight| (lines 4 to 10 above. Variable \verb|$_[0]| refers
to the parser object), which stores the current probability
distribution on a token probability stack,
increases the weigth for the closing parenthesis (line 7)
and decreases the weight for the open parenthesis (line 6).
This way we make more unlikely the presence of nested 
parenthesis and increase the certainty that the parenthesized expression
will be not too large.
We also increase the weight of token \verb|'+'| (line 8),
since  we humans,  ordinarily  make use of parenthesis to alter
the precedence of additive operators versus multiplicative 
operators. The final semantic action at lines 13-16 restores the former
probability distribution (line 14) and returns the concatenation of 
the three generated substrings (line 15).

Remember that we had - Listing \ref{listing:lexergen}, lines 4-12 - initially set 
the weight of token \verb|VAR| to zero so that at the beginning 
of the generation no \verb|VAR| tokens may appear.
This weight is increased as soon as the first assignment 
statement occurs (see line 4 below):
\begin{lstlisting}[%caption={The Grammar}, label={listing:parsergen2}, 
        %frame=bottomline,
        keywords={defaultaction,deltaweight,Paste,use, base, Gen, sub, pushdeltaweight,popweight,Unit,
                  my,generate,while,do, return}
        %firstnumber=1, 
        %frame=single,
        %numberstyle=\tiny,
        %numbers=right
        ]
 1 stmts:
 2     stmt
 3       {
 4         $_[0]->deltaweight(VAR => +1); 
 5         $_[1];
 6       }
 7   | stmts ';' { "\n" } stmt 
 8 ;
\end{lstlisting}

An intermediate action \verb| { "\n" } | is set at line 7 
for production \verb|stmt| $\rightarrow$ \verb|stmts| \verb|';'| \verb|stmt|
to trick the default action to insert a carriage return 
after each generated statement.

The semantic action for production \mbox{\tt stmt $\rightarrow$ VARDEF '=' exp}
- in the listing below - shows a way to
produce expressions free of  {\tt division by zero} and other floating point 
errors:

\begin{lstlisting}[%caption={The Grammar}, label={listing:parsergen2}, 
        %frame=bottomline,
        keywords={defaultaction,deltaweight,Paste,use, base, Gen, sub, pushdeltaweight,popweight,Unit,
                  my,generate,while,do, return}
        %firstnumber=1, 
        %frame=single,
        %numberstyle=\tiny,
        %numbers=right
        ]
 1 stmt:
 2     $VARDEF '=' $exp  
 3       {
 4         my $self = shift;
 5 
 6         my $res = EVALUATE($exp);
 7         return '' if $res =~ /^error/;
 8 
 9         $st{$VARDEF} = { exp   => $exp, 
10                          value => $res};
11 
12         "$VARDEF=$exp";
13       }
14 ;
\end{lstlisting}

Line 2 shows the use of the $dollar$ notation in eyapp. By 
prefixing the symbols \verb|VARDEF| and \verb|exp|  
with a dollar we can later refer 
inside the semantic action to their asociated
attributes by variables with the same name (lines 6 and 9),
instead of using the classical positional notation (a.k.a 
\verb|$_[1]| and \verb|$_[3]|).

At line 6 we evaluate the generated expression.
If there were {\tt division by zero} or other floating point errors
the expression is discarded (line 7).
Otherwise, the result and definition 
of this variable are stored in its corresponding
entry in the symbol table \verb|st| (lines 9-10).
Finally, the concatenation of the generated strings 
is returned at line 12.

\begin{lstlisting}[caption={The Grammar}, label={listing:parsergen2}, 
        frame=bottomline,
        keywords={defaultaction,Paste,use, base, Gen, sub, pushdeltaweight,popweight,Unit,
                  my,generate,while,do, return}
        %firstnumber=1, 
        %frame=single,
        %numberstyle=\tiny,
        %numbers=right
        ]
stmts:
    stmt
      {
        $_[0]->deltaweight(VAR => +1); 
        $_[1];
      }
  | stmts ';' { "\n" } stmt 
;

stmt:
    $VARDEF '=' $exp  
      {
        my $self = shift;

        my $res = EVALUATE($exp);
        return '' if $res =~ /^error/;

        $st{$VARDEF} = { exp => $exp, value => $res};

        "$VARDEF=$exp";
      }
;
exp:
    NUM                
  | VAR
  | exp '+' exp        
  | exp '-' exp        
  | exp '*' exp        
  | exp '/' exp        
  | exp '^' exp        
  | '('   { 
            $_[0]->pushdeltaweight(
              '(' => -1, 
              ')' => +1, 
              '+' => +1, 
            ); 
          } 
      exp 
    ')'
      {
         $_[0]->popweight; 
         "($_[3])"
      }
;

%%
\end{lstlisting}

The listing below shows the tail section.
The syntax generator is created in line 13.
The subsequent call to method \verb|LexerGen| 
sets the  lexical generator (line 14, expanded in Listing \ref{listing:parsergen1}).
Finally, inside the loop in lines 15-20, the 
call to \verb|generate| 
returns the generated expression
which is evaluated in line 17 and printed
in line 19. 

\begin{lstlisting}[caption={Using the Generator}, label={listing:usinggen}, 
        frame=bottomline,
        %frame=none,
        keywords={Int,String,YYParse,LexerGen,for,YYParse,generate, my, print, sub, pushdeltaweight,popweight,Unit}
        %firstnumber=1, 
        %frame=single,
        %numberstyle=\tiny,
        %numbers=right
        ]
 1 %%
 2 
 3 use Test::LectroTest::Generator qw(:all);
 4 
 5 sub EVALUATE { 
 6   ...
 7 }
 8 
 9 sub main {
10   my $package = shift;
11   my $numtimes = shift || 1;
12 
13   my $gen = $package->new(); 
14   $gen->LexerGen( ... );
15   for (1..$numtimes) {
16     my $exp = $gen->generate(); 
17     my $res = EVALUATE($exp);
18 
19     print "\n# result: $res\n$exp\n";
20   }
21 }
\end{lstlisting}

When compiled and executed 
\begin{lstlisting}[%caption={Generating a Generator of Arithmetic Expressions}, label={listing:parsergen}, 
        frame=none,
        keywords={eyapp, defaultaction,Paste,use, base, Gen, sub, pushdeltaweight,popweight,Unit,
                  my,generate,while,do, return}
        %firstnumber=1, 
        %frame=single,
        %numberstyle=\tiny,
        %numbers=right
        ]
$ eyapp -C GeneratorE.eyp 
$ ./GeneratorE.pm 
\end{lstlisting}

the program produces
outputs similar to this:
\begin{lstlisting}[%caption={Generating a Generator of Arithmetic Expressions}, label={listing:parsergen}, 
        frame=none,
        keywords={defaultaction,Paste,use, base, Gen, sub, pushdeltaweight,popweight,Unit,
                  my,generate,while,do, return}
        %firstnumber=1, 
        %frame=single,
        %numberstyle=\tiny,
        %numbers=right
        ]
# result: 3
C=6*1;
F=4;
F=8^(2/C+1/8/5)*F/7*1;
KK=F+2;
QL=KK;
J=6*0;
BA=KK/KK+2*J*KK+2
\end{lstlisting}

%\section{Computational Results}
%\label{section:computationalresults}
%\input{computationalresults.tex}


\section{Conclusions}
\label{section:conclusion}
We have presented a new algorithm to compute the exact list of tokens 
expected by the syntax analyzer at any point of the scanning process.
We have explained how this knowledge can be applied 
to solve the difficulties that arise  in
those languages where a token's type depends upon
contextual information and how
it can be used to derive generators 
of syntax and semantically correct data phrases.

This approach can deterministically parse and scan a class
of languages that is strictly larger than what is possible with
traditional LR parsers and disjoint scanners. The principle is
that when the scanner can be more discriminating in the tokens that
it returns it can help the parser to recognize a wider class of
languages. Since the parse state is used by the scanner to be more
discriminating, there is an effective cooperation between them.
\verb|eyapp| is freely available on the Internet at  \cite{Rodriguez:Leon}.

Aho, Sethi, and Ullman \cite[p.84-85]{aho1986}, 
Aycock and Horspool
\cite{schrodingerstoken} and 
Wyk and Schwerdfeger \cite{wyk}
among others,
mention several reasons why it is best to kept apart parser
and scanner.
A disjoint parser and scanner should be used for languages in which
this is possible. The disadvantages resulting of not tying them 
togheter can be overcomed via interactions between the two components as 
the one presented in this contribution.


\section*{Acknowledgments}
This work has been supported by the \textsc{ec} (\textsc{FEDER}) and
the Spanish Ministry of Science and Innovation inside the 'Plan Nacional de
\textsc{i+d}+i' with the contract number \textsc{tin2008-06491-c04-02}.
%
It has also been supported by the Canary Government project number
\textsc{pi2007/015}.



\begin{thebibliography}{99}
\input{ppcr_ciaa_bib.tex}
\end{thebibliography}


\end{document}




